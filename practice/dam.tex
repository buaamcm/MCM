\documentclass{article}
\usepackage{CJK}
\begin{CJK}{UTF8}{song}
\begin{document}

%水电站设计,建造费用的估算
水电站设计费用$C_{d}$是工程费用$C_{p}$(按美元计)的函数,通常可用$C_{d} = 0.34C_{p}^{0.9}$来表示。也可以用水电站装机容量$V(kW)$ 及其设计水头$H(m)$的关系来表示;如式
\[C_{p} = K\left(\frac{V}{\left(\frac{H}{0.3}\right)^{0.3}}\right)^{0.82}\]
将$C_{p}$值代入$C_{d}$式中,则
\[C_{p} = 0.34K^{0.9}\left[\frac{V}{\left(\frac{H}{0.3}\right)^{0.3}}\right]^{0.74}\]

通常大型水电站$K = 7.7\times10^{4}$,小型水电站$K = 5 \times 10^{4}$,若有现存已建成的水坝,上述$K$值应需乘上0.55

编制水电站项目的可行性报告费用,根据世界银行统计,可行性报告编制费用约为水电站设计费用的10\%,而前期可行性报告编制费用则为水电站实际费用的2\%

具体情况可能需要在上述公式基础上乘上$0.7\sim1.3$

%题目中没有要求说水坝必须是水电站,如果仅仅是建造水坝的成本估计还没有找到。。。,但可以从上述的0.55估计出,仅修建水坝的K小于等于0.45

%下面的部分是简要评价的初稿,从气候变化,设计、建设成本,生态成本,水资源的调度,紧急情况处理,极端情况应对,区域影响方面讨论
(对Kariba Dam面临问题的严重性的分析放在Summary里,这里就不重复了?)解决Kariba Dam问题的方式可以简单地分成3 类,即修复水坝,重建水坝和移除水坝并以其他水坝代替,对于第类解决方式,ZRA 对于新建$10\sim20$个小水坝的方案感兴趣。
对三种方案从成本和收益角度进行评估是一件复杂的任务,因为它们受到诸多因素的影响,其中一些因素的变动幅度较小,例如用设计水头和装机容量估算的设计、建设费用。另一些因素的估计则有较大的不确定性,例如未来的气候变化对水利发电带来的影响:温度升高导致蒸发量增大,更重要的是,降雨量可能发生变化并影响入库流量,这类因素的影响可能比容易准确估计的设计、建设成本因素有更为深远的影响。(上面两个例子一个是说的成本,一个说的是收益,可能换成下面的说法更好)

对三种方案从成本和收益角度进行评估是一件复杂的任务,因为它们受到诸多因素的影响。仅就成本而言,虽然修建水电站的成本可以使用公式
\[C_{p} = K\left(\frac{V}{\left(\frac{H}{0.3}\right)^{0.3}}\right)^{0.82}\]
较为准确地进行估计,其中$C_{p}$为水电站建设成本,$V$为装机容量,$H$为设计水头,$K$为比例系数;但是建造水坝的生态成本则需要更谨慎的考虑,因为对生态环境的破坏可能是不可逆的。

第一种方案,修复水坝。建设成本最低的方案,同时,因为不会改变流域被淹没的范围,所以没有额为的生态成本。从收益角度分析,修复的同时可以进行Kariba Dam水电站的扩建,能有效提高水电站的总装机容量,从而提高水电站的收益。实际上,对Karibra Dam水电站的扩建工程正在进行。由于修复(完成后)不对Karibra湖造成影响,所以原本从湖中取水使用得到的收益不会减少。以上的分析建立在假定气候不发生剧烈变化和不发生历史统计外的紧急状况的基础上。

第二种方案,重建水坝。水坝的重建设计原有水坝的拆除,这就增加了工程的技术难度和风险,并且,因为水坝的重建必然导致水电站较长时间段内不能发电,所以这部分收益的损失也应该计入重建的成本。但是,除了更有利于扩充水电站装机容量(重新设计更合理的发电机安放方式和使用更先进的设备替代原有的设备)之外,重建水坝也可以伴随修复水坝难以具有的好处。新建的水坝可以具有更强的抗洪能力,这将使河流管理部门能更灵活的处理紧急情况,并且,更强的蓄水能力也将提高水电站的发电能力,当然,提高Karibra湖的水位伴随生态风险,所以需要谨慎对待

第三种方案,移除水坝并以$10\sim20$个小水坝替代。雄心勃勃的计划。根据上文中的建设成本估计公式,即使所有小水坝的装机容量之和与Karibra Dam相同,总建造成本预计也会高于重建Karibra Dam的成本。(这里我假设咱们对requirement 2 的解答在原Kariba Dam的位置替代以小水坝)与重建计划相同,移除Karibra Dam必然导致发电收益上的损失,并且,即使是在Karibra Dam原位置新建小水坝也可能导致蓄水量上的损失,因为Karibra湖的水位会下降;但这些可以通过合理的规划使得损失降低到最小,具体来说,可以优先新建小水坝,逐步替代Karibra Dam的发电能力;在下游新建水坝以存续移除Karibra Dam导致的Karibra湖损失的水量,以降低总蓄水量的损失。新水库淹没区域的经济补偿也需要计入成本之内(这里我们可以通过计算流域的单位面积GDP做出一个估算),另外,Karibra湖水位变化导致的周边设施(码头)和相关行业(渔业)的损失也需要计入成本。从生态环境角度,第三种方案同样伴随着更大的风险,因为它不仅将淹没新的区域,还会影响到Kariba湖的生态(湖水水位下降,湖分割成几个部分)。收益方面,水坝系统间水资源的调度会减少泄洪导致的水资源的损失,实际将有助于提高水电站的发电量,并且,将蓄洪量在水坝之间合理分配将提高水坝系统的安全性,并且缩小的水库面积将减小水的蒸发损失,在面临紧急情况时,河流管理部门也可以采取更加灵活的应对方案。鉴于第三种方案的高成本,更加长远的分析是有必要的。未来,气候变化可能使得赞比西河流量下降$40\%\sim50\%$,当然,当前的气候预测带有较大的不确定性,但是,确实不应对新的水坝系统的收益盲目乐观。
(暂时的思路是,成本从建设、减少发电、损失水量、环境、相关设施损失、相关行业损失方面考虑,收益从发电,安全性,紧急情况,气候方面考虑)

\end{CJK}{UTF8}{song}
\end{document}
