\documentclass{article}
\usepackage{CJK}
\begin{CJK}{UTF8}{song}
\begin{document}

%问题描述
对于一个给定水坝系统,一系列水坝的选址和相应的库容、装机容量、当地降水量等限制都已经可以确定。然而,仍然有很多分析可以开展,其中,最引人关注的是水坝系统间水资源的调度方案。调度方案能够保障安全性并且可应对紧急情况(洪水和最小水流量),并且能够尽可能的保护水资源和从中得到的收益。可以预料到,在不同的情况下(水周期中的不同时段,水库的不同蓄水情况等),调度方案将是不同的。为了合理的应对不同情况,需要建立一个生成水资源调度方案的模型。\\

%符号
$V$ 水资源体积 \\
$V_{D}$ 最终整个水坝系统因为泄洪而损失的水体积 \\
$V_{D,i}$从上游开始记的第$i$个水库在单位时间段内泄洪的水体积 \\
$\triangle V_{i}$  单位时间水库中水量的变化 \\
$V_{n, i}$ 单位时间内水库$i$水量的自然增量,即单位时间内因为自然原因(降水,支流汇入等)汇入的水量减去时间段内的蒸发损失 \\
$V_{e, i}$ 单位时间水库$i$发电使用的水量 \\
$V_{u, i}$ 单位时间水库$i$用于其他用途的水量 \\
$n$ 水坝系统中的水坝数量,所以,系统中也有$n$座水库 \\


%模型假设
1、水资源的利用分为多种,例如水利发电、农业灌溉等,为了简化水资源价值的评估,我们假设水资源的价值与水资源的体积$V$成正比 \\
2、水坝系统因为泄洪而损失的水量不能再利用 \\
3、发电用水汇入下游水库,但其他用途用水不直接汇入水坝系统 \\

%模型建立
根据我们的前两条假设,单位时间内,水坝系统损失的水资源价值正比于因为泄洪而损失的水量。而水只有从水坝系统的最后一个水坝被以泄洪的方式排出时,才是从整个水坝系统中损失了,所以有
\begin{equation}
V_{D} = V_{D, i}
\end{equation}
那么,想要尽可能提高水坝系统的收益,就需要是的$V_{D, i}$尽可能的小,同时,不同情况下的限制性因素也必须在提出调度方案时被考虑到。于是,我们考虑使用优化模型来得到一定情况下的水坝系统的水资源调度方案。\\

根据我们上面的论述,优化的目标函数是非常简单的,即:
\begin{equation}
\min V_{D} = V_{D, i}
\end{equation}

下面,我们将焦点放到寻找约束条件上:\\
首先,单位时间后水库中的水量等于原水量,水库自然增量,上游水库排水(包括发电用水和泄洪水量)之和减去其他用途用水,发电用水,泄洪水量之和,即:
\begin{equation} V_{i}^{'} =
\left\{
\begin{array}{cc}
 V_{i} + V_{n, i} + V_{D, i - 1} + V_{e, i - 1} - V_{u, i} - V_{e, i} - V_{D, i} & i > 1 \\
V_{i} + V_{n, i} - V_{u, i} - V_{e, i} - V_{D, i} & i = 1 \\
\end{array}
\right. 
\end{equation}
其中,$V_{i}^{'}$为时间段末水库$i$中水量, 记$\triangle V = V^{'} - V$,则有:
\begin{equation}
V_{n, i} = 
\left\{
\begin{array}{cc}
\triangle V_{i} - V_{D, i - 1} - V_{e, i - 1} + V_{u, i} + V_{e, i} + V_{D, i} & i > 1 \\
\triangle V_{i} + V_{u, i} + V_{e, i} + V_{D, i}& i = 1 \\
\end{array}
\right.
\end{equation}
这是变量间的等式约束关系\\

%后面会有更多符号

我们考虑对变量的更多约束:
大坝的发电能力是有限的,相应的,发电消耗的水量也是有限的;并且,发电的用水量显然不能是负值,所以:
\begin{equation}
0 \leq V_{e, i} \leq V_{e, i}^{\max}
\end{equation}
类似的,对于$V_{u, i}$有:
\begin{equation}
V_{u, i}^{\min} \leq V_{u, i} \leq V_{u, i}^{\max}
\end{equation}
考虑到水库周边的农业用水等有刚性需求,所以$V_{u, i}$的下限不是$0$ \\ 
对于$\triangle V_{i}$的约束应取决于水库的现有水量,对水库水量的未来计划以及未来水库水量的自然补充;但直观上,水库的水位不应有剧烈的波动;并且我们的模型在具体使用时应允许相关领域的专业人士给出水库水位波动的限制,所以我们将对$\triangle V_{i}$的约束写作:
\begin{equation}
\alpha_{i} \leq \frac{\triangle V_{i}}{V_{i}^{\max}} \leq \beta_{i} 
\end{equation}
其中$V_{i}^{\max}$为水库的最大容积\\
除上述以外,考虑到需要满足流域和其他区域的用电需求,大坝系统的总发电量应有一个大于$0$的下限,虽然由于存在电网的调节,系统中的某一些大坝可以在某时段不发电,所以有:
\begin{equation}
\sum_{i = 1}^{n} V_{e, i} \geq E_{min}
\end{equation}
$E_{min}$是产生最小用电需求电能所需的水量

%算法没有放到格式里

\end{CJK}{UTF8}{song}
\end{document}