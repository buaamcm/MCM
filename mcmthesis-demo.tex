%%
%% This is file `mcmthesis-demo.tex',
%% generated with the docstrip utility.
%%
%% The original source files were:
%%
%% mcmthesis.dtx  (with options: `demo')
%% !Mode:: "TeX:UTF-8"
%% -----------------------------------
%%
%% This is a generated file.
%%
%% Copyright (C)
%%     2010 -- 2015 by latexstudio
%%     2014 -- 2016 by Liam Huang
%%
%% This work may be distributed and/or modified under the
%% conditions of the LaTeX Project Public License, either version 1.3
%% of this license or (at your option) any later version.
%% The latest version of this license is in
%%   http://www.latex-project.org/lppl.txt
%% and version 1.3 or later is part of all distributions of LaTeX
%% version 2005/12/01 or later.
%%
%% This work has the LPPL maintenance status `maintained'.
%%
%% The Current Maintainer of this work is Liam Huang.
%%
\documentclass{mcmthesis}
\mcmsetup{CTeX = true,   % 使用 CTeX 套装时,设置为 true
        tcn = 57868, problem = A,
        sheet = true, titleinsheet = true, keywordsinsheet = true,
        titlepage = true, abstract = true}
\usepackage{palatino}
\usepackage{caption}
\usepackage{mwe}
\usepackage{amsmath}
\usepackage{lipsum}
\usepackage{times}
\usepackage{mathptmx}
\usepackage{caption}
\usepackage{subfigure}

\title{Managing The Zambezi River}
\author{Kai Feng, Song Lu, Yutao Zeng}
\date{\today}
\begin{document}
\begin{abstract}
Here is the abstract to be written!
\begin{keywords}
Kariba Dam; Multi-dams arrangement;
\end{keywords}
\end{abstract}
\maketitle
\section{Introduction}
\indent \indent The Kariba Dam is one of the biggest dam in the world, which is constructed on the Zambezi River. It supplies 1626 megawatts of electricity to parts of both Zambia and Zimbabwe. However, the Kariba Dam is in a dangerous state now. In the past 50 years, the torrents from the spillway have eroded its bedrock, carving a vast crater that has undercut the dam's foundations.$^{[1]}$ A number of options are available to solve this problem. This paper focuses on the third option -- Removing the Kariba Dam and replacing it with a series of small dams along the Zambezi River. To find the best location for new dams and do the best arrangement with the multi-dam system, we propose two mathematical models. This paper describe these two models minutely and give suggestions on where to build new small dams and how to dispatch all the dams. \\

%%%%%%%%%%%%%%%%%%%%%%%%%%%%Introduction ends here%%%%%%%%%%%%%%%%%%%%%%%%%%%%%%

\section{Model A: Search for possible locations to build dams}
\subsection{Description}
\indent \indent To build new dams, we need to find some proper locations at first. However, we cannot pick all the suitable locations manually since Zambezi River is rather long. So our underlying idea is fairly simple. Firstly, we find a serial of possible river reaches. Although only a rough estimate, it does help us to exclude many reaches which cannot meet the requirements. Then we can pick some suitable locations from the reaches left. In this step, we need to refine this problem. Generally, the choice of dam's location should be related to geology, terrain, economic, ecology, disaster and other factors. Among all the factors, the dominant factor should be terrain for it decides both the safety and the economy of the reservoir. We build a formula as a benchmark to quantify their impact to the selection and pick the locations with the highest grade as our result. The following is the detailed discussion. \\

\subsection{Analysis and Assumptions}
\indent \indent To be specific, we get two principle of searching for possible reaches to build dams:
\begin{enumerate}
  \setlength{\itemsep}{0pt}
  \setlength{\parsep}{0pt}
  \setlength{\parskip}{0pt}
  \item The higher the vertical drop is, the more abundant hydropower resource is contained;
  \item In consideration of reducing ecological impact and water evaporation, the surface area of reservoirs should be small under certain requirement of volume. To build reservoir with small surface area and certain volume, the average depth of reservoir should be deep, thus, dams should be built between deep ravines.
\end{enumerate}

\indent Hydropower station convert gravity potential of water into electrical energy. The gravity potential is calculated as $E_{p} = mgh$, thus, higher vertical drop means bigger electricity-generation capacity. In order to simplify the expression, all symbol used in this model are listed in the table below.\\

\begin{table}[!ht]
\centering
  \begin{tabular}{cc}
  \hline
  Symbols & Meanings \\
  \hline
  $V$ & the volume of reservoir \\
  $S$ & the water surface area of reservoir \\
  $H$ & the water depth of reservoir \\
  $\alpha$ & the angle of bank slope \\
  $g$ & the gravitational acceleration \\
  \hline
  \end{tabular}
\caption{Symbol Table of Model A}
\end{table}

\indent To simplify model, we assume that the vertical section of a reservoir is an approximate trapezoid, then the submerged area can be expressed as below:
\begin{equation}\int_{A}^{B}\frac{H\left(l\right)}{\sin\left[\alpha\left(l\right)\right]}dl + \int_{C}^{D}\frac{H\left(L\right)}{\sin\left[\alpha\left(L\right)\right]}dL + S_{bottom}\end{equation}
where $l, L$ are respectively the lengths of left bank and right bank; $A, B$ indicate the starting position and end position of $l$; similarly, $C, D$ indicate the starting position and end position of $L$.
The expression $\left(1\right)$ is still hard to use because of the difficult estimation of $\alpha$. In order to make our assessment feasible, we need to simplify the expression $\left(1\right)$. Noticed that:
\begin{equation}
\int_{A}^{B}\frac{H\left(l\right)}{\sin\left[\alpha\left(l\right)\right]}dl = \frac{1}{sin\left[\alpha\left(\zeta\right)\right]}\int_{A}^{B}H\left(l\right)dl
= KH_{average}l
\end{equation}
where $K$ is a coefficient of inclination. expressions $\left(2\right)$ is a application of mean value theorem for integrals, it fits in with the physics intuition. Then, the submerged area can be estimated as:
\begin{equation}
\left(K_{1}l + K_{2}L\right)H_{average} + S_{bottom}
\end{equation}
Using expression $\left(3\right)$, we can qualitatively explain why small surface area is needed. Using equation $H_{average} = \frac{V}{S}$, we get:
\[S_{submerged} \approx \left(K_{1}l + K_{2}L\right)\frac{V}{S} + S_{bottom}\]
and using the assumption of vertical section, the area of the bottom of a reservoir can be estimated as:\\
\begin{equation}
S_{bottom} = \int\left(\beta dS \right) \propto S
\end{equation}  
where $\beta$ is a coefficient of water surface area size and bottom area size, the expression $\left(4\right)$ qualitatively explain that $S_{bottom}$ is proportional to $S$, thus we get:\\
\begin{equation}
S_{submerged} \approx \left(K_{1}l + K_{2}L\right)\frac{V}{S} + CS
\end{equation}
where $C$ is a coefficient to indicate that $S_{bottom}$ is proportional to $S$.\\
\indent Since the right side of the expression $\left(5\right)$ is a hyperbolic function, it monotonically decrease when $S \geq \sqrt{\frac{\left(K_{1}l + K_{2}L\right)V}{C}}$. In the actual situation, $S \gg \sqrt{V}$, so we can qualitatively conclude that under certain requirement of volume small surface area of reservoir is more beneficial than the bigger surface area.\\
\indent According to the discussion above, we should find the reaches with big throws on the Zambezi river based on the first principle. In accordance with the second principle, the possible reaches should between deep ravines, because a reservoir in deep ravines can have deeper water depth and thus smaller surface area.


\subsection{Model Building}
\subsubsection{Search for candidate areas}
\indent \indent In order to find suitable dam sites along the Zambezi River, we established a simple model based on the geographical conditions, costs, safety of the whole dam system and other fatal conditions. \\
\indent The storage capacity of dams depends on the height of dams which is limited by the slope and height of the river bank. The cost of construction basically depends on the dam's height and length as well as the width of river. Thus, there are three major parameter to be taken into account in choosing candidate regions:
\begin{itemize}
\item The vertical drop of the river.
\item The slope and height of the river bank.
\item The width of river.
\end{itemize}

We download the geomorphological data of the Zambezi River Basin and generate a Digital Elevation Model (DEM). The Figure 1 is a general overview of the elevation in that region (the Zambezi River is marked in red line in the chart).

\begin{figure}[h]
\small
\centering
\includegraphics[width=14cm]{./figures/Sensing_Figure.png}
\caption{Overview Elevation Chart} \label{fig:Fig1}
\end{figure}

By using DEM, we obtain the elevation along the whole Zambezi River and plot the elevation figure as Figure 2. From Figure 2, We can obviously notice that the river is divided into 3 parts. There is a clear trend that the upstream is relatively plain, the water level decreases remarkably in the midstream and shoulders the most responsibility of storing water, the downstream have a rapid change of water level as well but there are few dams. Three prominent falls of the elevation are the Victoria Fall, the Kariba Dam and the Cahora Bassa Dam.

\begin{figure}[h]
\small
\centering
\includegraphics[width=14cm]{./figures/dis_alti_v3.png}
\caption{Elevation along the Zambezi River} \label{fig:Fig2}
\end{figure}

%%%%%%%%%%%%%%%%%attention here!%%%%%%%%%%%%%%%%%%%%%%%%%%
From the perspective of the vertical drop, we know the following areas are suitable for the establishment of dams: a few areas of upstream, reaches between Victoria Fall and Kariba Lake, reaches between Kariba Dam and Cahora Bassa Dam, some areas in the downstream. The exact areas are marked in red in Figure 3. In addition to the drop, we also need to analyze the slope and height as well as the breadth of the Zambezi River itself for they have great affect on the cost of dam construction. We plotted the contours of the Zambezi River basin based on the DEM model above. Generally, for the sake of storage capacity, safety and ecological impact, the bank of the reservoir should be steep as possible and have suitable height. The steepness of the bank means less flooded area and less impact on the surrounding environment. We also know that the dam should never be higher than the bank, so considerable height of the bank allows the reservoir to have a higher water level and proofs the robustness of the dam in extreme cases like flood. In addition, we don't expect to build our dam across a wide river. Not only because building dams in narrow valley can significantly reduce construction cost, but also since the too long dam body may have a negative impact on the safety of the dam.\\
\indent We plot the contours of Zambezi River basin by using the DEM model above. The followings are 2 samples of the whole picture whose contour interval is 5 meters. The dense the contours means the river bank is steep which is good for us to build the dam.\\

\begin{figure}[h]
\small
\centering
\includegraphics[width=8cm]{./figures/highlight1.png}
\caption{Candidate areas according to vertical drop} \label{fig:Fig3}
\end{figure}

\begin{figure}[htbp]
\centering %居中
\subfigure[Instance with dense contours]{ %第一张子图
\begin{minipage}{7cm}
\centering %子图居中
\includegraphics[scale=0.22]{figures/midstream_contour.png} %以pic.jpg的0.5倍大小输出
\end{minipage}
}
\subfigure[Instance with sparse contours]{ %第二张子图
\begin{minipage}{7cm}
\centering %子图居中
\includegraphics[scale=0.3]{figures/upstream_plain_contour.png} %以pic.jpg的0.5倍大小输出
\end{minipage}
}
\caption{Contour samples along Zambezi River} % %大图名称
\label{fig:1.3.1} %图片引用标记
\end{figure}
\indent Taking these factors into account, we further reduced the candidate areas as Figure 5.
\begin{figure}[h]
\small
\centering
\includegraphics[width=10cm]{./figures/dis_alti_v3.png}
\caption{Candidate areas according to combined factors} \label{fig:Fig4}
\end{figure}
\indent In addition, we need to consider the geological foundation conditions on both sides of the river. The rock layer of the candidate areas are rather solid and the vegetation along the river is lush which means it is less like to happen landslide in these areas. Base on the discussion above, we don't narrow the candidate areas any more since they are suitable for the construction of dams, the next step is build a function as benchmark and pick the exact locations from these candidate areas.

\subsubsection{Building Benchmark function}
\indent \indent From the discussion above, we have obtained the candidate dam construction area. However, evaluate from the continuous area is a heavy load work with great difficulty and complexity, so we try to turn the continuous problem into discrete problem and reduce the complexity of the problem.\\
\indent The choose of the dam should based on various factors, such as vertical drop, rock and soil condition, the slope and height of the bank and the width of the river etc. With so many factors to be considered, we decided to use analytic hierarchy process(AHP) to do this job.\\
\indent Firstly, we need to establish the natural coordinate along the river. We assume $l$ to be the distance from the region's starting point the the candidate point.

\begin{table}[!ht]
\centering
  \begin{tabular}{cc}
  \hline
   Intensity of Importance & meaning  \\
  \hline
  $1$ & Equal importance \\
  $3$ & Moderate importance \\
  $5$ & Strong importance \\
  $7$ & Very strong or demonstrated importance \\
  $9$ & Extreme importance \\
  $2,4,6,8$ & Intensity between two hierarchies \\
  $1,1/2,...,1/9$ & The opposite numbers\\
  \hline
  \end{tabular}
  \caption{Intensity of Importance}
\end{table}

We combined these factors into a hierarchical factor vector $\textbf{P}_1$, and get $\textbf{P}_1 = (F_{l}, \alpha_{l}, H_{l}, C_{l}, W_{l})$. Then we can get the comparison matrix $\textbf{A}_1$ which is satisfied with the following equation:
\begin{equation}
\textbf{A}_1 = (a_{i,j})_{5\times5},\qquad a_{ij} > 0, \quad a_{ij} = \frac{1}{a_{ji}}
\end{equation}
Now we need to get the importance of each factor, that is , the weight vector $\textbf{w}$. The $C_{l}$ and $W_{l}$ decides whether the construction is feasible to a great extent, so they are regarded as the most important factors. Between $W_{l}$ and $C_{l}$, $W_{l}$ has more flexibility, so we decide to choose $W_{l}$ as a more effective judge factor. In the other 3 elements, $F_{l}$ and $H_{l}$ decide they storage capacity of the dam, so they are also with a certain importance. However, $\alpha_{l}$ mainly influence the submerged area, so we think it a less significant factor. So we get the comparison matrix as below:
\[\textbf{A}_1 = 
\left[
\begin{matrix}
1 & 5 & 1 & \frac{1}{3} & \frac{1}{4} \\
\frac{1}{5}  & 1 & \frac{1}{5} & \frac{1}{7} & \frac{1}{8} \\ 
1 & 5 & 1 & \frac{1}{3} & \frac{1}{4} \\
3 & 7 & 3 & 1 & \frac{1}{2} \\
4 & 8 & 4 & 2 & 1 \\
\end{matrix}
\right]
\]
%%%%%%%%%%%%%%%%%%%%%%%%%%%%%%%%%%%attention here%%%%%%%%%%%%%%%%%%%%%%%%%%%%%%%%%%%%%%%%%%%%%%
%%%%%%%%%%%%%%%%%%%%%%%%%%%%%%%%consistency index%%%%%%%%%%%%%%%%%%%%%%%%%%%%%%%%%%%%%%%%%%%%%%
Calculate the maximum eigenvalues of matrix $A$, we get $\lambda = 5.1415$. The consistency indicators of $\textbf{A}_1$ is \[\mathit{CI} = \frac{\lambda - n}{n - 1} = 0.0354\], where $n$ is the order of the matrix. When $n = 5$, the stochastic consistency index \textit{RI} is $1.12$. Then the consistency ratio of matrix A is \[\mathit{CR} = \frac{\mathit{CI}}{\mathit{RI}} = 0.0316 < 0.1 \]

\subsection{Model Validation}

\subsection{Result}
\indent \indent According to Model A, We get some important equations as follow.

Meanwhile, based on the model, we make the following suggestions.
\begin{itemize}
  \item 
\end{itemize}









\section{Model B : The Best Arrangement}
\subsection{Description}
\indent For a given dam system, a series of dam site selections and corresponding reservoir capacities, installed capacity, local precipitation limits, and so on, can already be determined. However, there are still many analyzes that can be carried out, among which the most interesting is the modulating of water resources between dams. The modulating scheme can protect the water resources and the benefits derived therefrom as much as possible on the basis of ensuring safety and responding to emergencies. It can be expected that the modulating scheme will be different in different situations (different periods in the water cycle, different water capacity of reservoirs, etc.). In order to cope with the different situations reasonably, a model of generating water resources modulating scheme is needed.\\

\subsection{Symbols}

\begin{table}[h]
\centering
  \begin{tabular}{cc}
  \hline
  Symbols & Meanings \\
  \hline
  $V$ & Water Resources Volume \\
  $V_{D}$ & The volume of water lost to the dam system as a result of flood discharge \\
  $V_{D,i}$ & The volume of water discharged from the reservoir $i$ during the unit time period.\\
  $\triangle V_{i}$ & Variation of Water Quantity per Unit Time in Reservoir $i$\\
  $V_{n, i}$ & The natural water increase in reservoir $i$ per unit time \\
  $V_{e, i}$ & The amount of water used per unit time for power generation of reservoir $i$\\
  $V_{u, i}$ & The amount of water in reservoir $i$ used for other purposes per unit of time \\
  $n$ & The number of dams in the Dam System, so the system also has $n$ reservoirs \\
  \hline
  \end{tabular}
\caption{Symbol Table of Model B}
\end{table}

\subsection{Assumptions}
\begin{enumerate}
  \item The use of water resources is divided into a variety, such as hydropower, agricultural irrigation. In order to simplify the assessment of water resources value, we assume that the value of water resources is proportional to the volume of water resources $V$. 
  \item The amount of water lost to a dam system due to flood discharge can not be reused.
  \item The power generation water is transferred into the downstream reservoir, but the water for other uses is not directly transferred into the dam system.
\end{enumerate}

\subsection{Model Building}
According to our first two assumptions, the value of water lost per unit time of a dam system is proportional to the amount of water lost due to flood discharge. 
\subsection{Optimization}

\subsection{Sensitivity Analysis}

\subsection{Model Validation}

\subsection{Result}

\section{Strengths and weaknesses}
\subsection{Model A: Search for possible locations to build dams}
Model A has the following weaknesses.
\begin{itemize}
\item 
\item 
\end{itemize}
Despite of the weakness, it has more strengths.
\begin{itemize}
  \item 
  \item 
\end{itemize}

\subsection{Model B: The Best Arrangement}
In Model B, we consider .... , which is a weakness. However we have more strengths as below:
\begin{itemize}
  \item 
\end{itemize}

\section{Conclusion}
\indent \indent Here is the conclusion to be done!

\begin{thebibliography}{99}
\bibitem{1} IRMSA , Impact of the failure of the Kariba Dam, June 2015.
\bibitem{2}Lamport, Leslie,  \LaTeX{}: `` A Document Preparation System '',
Addison-Wesley Publishing Company, 1986.
\bibitem{3}\url{http://www.latexstudio.net/}
\bibitem{4}\url{http://www.chinatex.org/}
\end{thebibliography}

\clearpage
\section{Brief Assessment of the options}
\indent \indent The solution to the Kariba Dam problem can simply be divided into three options: repairing it, rebuilding it or removing it then replacing it with other dams. To the third method, ZRA suggests to build $10\sim20$ small dams to replace the huge Kariba Dam.\\
\indent Evaluating the options from the perspective of cost and benefit is a complex task, since it can be influence by a number of factors. Only considering the cost of building dams, although it can be estimated accurately by using the cost formula below \\
\[
C_{p} = K\left(\frac{V}{\left(\frac{H}{0.3}\right)^{0.3}}\right)^{0.82}
\]
where $C_{p}$ is the cost of building the hydropower station, $V$ is the installed capacity, $H$ is the design head, $K$ is the proportional coefficient. However, the ecological costs of dam construction need to be considered more cautiously because damage to the ecological environment may be irreversible.\\
\indent Option 1. Repairing the existing Kariba Dam. This is the option with the lowest cost of construction. Meanwhile, it won't change the submerged area, so there is no extra ecological cost. From the aspect of revenue, the reconstruction and expansion of Kariba Dam hydropower station can be carried out at the same time, which can effectively increase the total installed capacity of hydropower station, and thus improve the income of hydropower station. In fact, the expansion of the Kariba Dam hydropower station is underway. Since the reconstruction will not affect the Kariba Lake, the benefits from the use of water from the lake won't be reduced. The analysis above is based on the assumption that the climate will not change drastically in the future and no rare disasters which is outside the historical statistics will occur.\\
\indent Option 2. Rebuilding the existing Kariba Dam. Because rebuilding the Kariba Dam need to remove the existing the dam and rebuild it at the origin site, it is an option with high risk and cost. What's more, the reconstruction of the dam will inevitably lead to the result that the hydropower station can't generate electricity in quite a long period, so this part of loss should also be included in the cost of reconstruction. However, rebuilding dams do have benefits. It helps to expand the installed capacity of hydropower station (benefit from re-designing the internal structure and using more advanced equipment). The new designed Dam would have better flood protection capacity, which allows river management to handle emergency with more flexibility. Stronger water storage capacity means we can raise the water level of Kariba Lake. It will increase the energy generation as well as bring the risk of ecologic damage which needs to be treated with caution.\\
\indent Option 3. Removing the Kariba Dam and replacing it with a series of $10\sim20$ smaller dams along the Zambezi River. This is quite an ambitious plan. Even if the sum of installed capacity of all these small dams is the same as that of Kariba Dam, the total construction cost is still expected to be higher than rebuilding Kariba Dam according to the cost formula above. With the same problem as option 2, removing Kariba Dam will definitely lead to the loss of energy generation, furthermore even the construction of a smaller dam in the original position of Kariba Dam may result in loss of water storage capacity, as the water level in Kariba Lake will decrease. Fortunately, these losses can be minimized through rational planning. Specifically, we can give priority to the construction of small dams, and then gradually replace the Kariba Dam with their power generation capacity. New dams built in the down stream would store the water from Kariba Dam when it is removed, which can reduce the loss of water resources. Different from the previous two options, economic compensation of the new reservoirs' reserved area also needs to be include in the cost.(Here we can make an estimate by calculating the unit area GDP of the catchment)From the ecological point of view, the third option is also accompanied by greater risk. It will not only flood new areas, but also affect the ecology of Lake Kariba (the water level drops and the lake is divided into several parts). In terms of revenue, the scheduling of water resources between dams will reduce the loss of water resources caused by flooding discharge, which will actually help to increase the power generation capacity of hydropower stations. Moreover, the rational allocation of flood storage between dams will increase the safety of the dam system, the reduced reservoir area will reduce the evaporation loss of water and, in the face of emergencies, river management can also adopt a more flexible approach. Because of the high cost of the third option, a long-term analysis is of great significance. In the future, the flow of Zambezi River may reduce by $40\%\sim50\%$ due to the climate change. Although the climate predictions nowadays are with a large degree of uncertainty, but we should never be blindly optimistic about the benefits of the new dam system.\\




% \clearpage
% \begin{appendices}

% \section{First appendix}

% \lipsum[13]

% Here are simulation programs we used in our model as follow.\\

% \textbf{\textcolor[rgb]{0.98,0.00,0.00}{Input matlab source:}}
% \lstinputlisting[language=Matlab]{./code/mcmthesis-matlab1.m}

% \section{Second appendix}

% some more text \textcolor[rgb]{0.98,0.00,0.00}{\textbf{Input C++ source:}}
% \lstinputlisting[language=C++]{./code/mcmthesis-sudoku.cpp}

% \end{appendices}

\end{document}